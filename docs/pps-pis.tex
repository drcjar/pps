%based on imperial example provided + hra example at http://www.hra-decisiontools.org.uk/consent/examples.html 
\documentclass[a4paper,10pt]{article}

%set font to Arial
%\usepackage{fontspec}
%\setmainfont{Arial}
\usepackage{helvet}
\renewcommand{\familydefault}{\sfdefault}

%graphics
\usepackage{graphicx}
%\usepackage{subfigure}
\usepackage{pslatex}
\usepackage{pstricks}

%math equations
\usepackage{amsmath}

%python code
%\usepackage{minted}

%headers
\usepackage{lastpage}
\usepackage{fancyhdr}
\pagestyle{fancy} 
\lhead{IRAS Project ID: 203355}
\chead{}
\rhead{clinicaltrials.gov: NCT03211507}
\lfoot{PPS}
\cfoot{Participant Information Sheet v0.1 \hspace{1cm} 7th August, 2017}
\rfoot{\thepage\ of \pageref{LastPage}}


%display URLS
\usepackage{url}

%hyperlinks
\usepackage{hyperref}

%comments
\usepackage{verbatim} 

%nice tables
\usepackage{booktabs}
\newcommand{\ra}[1]{\renewcommand{\arraystretch}{#1}}

%multi rows for the nice tables
%\usepackage{multirow} 

%nice diplay of code
%\usepackage{minted}

%nice references
\usepackage[super]{natbib}

%some maths
\usepackage{amsmath}

%margins
%\usepackage{geometry}
%\geometry{verbose,a4paper,tmargin=60mm,bmargin=25mm,lmargin=25mm,rmargin=25mm}

%in line citations
%\usepackage{bibentry}

%\hyphenpenalty=10000

%\nobibliography*

\newpage\title{\bf Participant Information Sheet} 
\date{}
%\author{Carl Reynolds \\
%\small National Heart \& Lung Institute, Imperial College London }

\begin{document}
\pagestyle{fancy} 


\pagenumbering{arabic}

%\pagestyle{empty}

%\maketitle

\section*{Participant Information Sheet}

\subsection*{Idiopathic Pulmonary Fibrosis Job Exposure Study (PPS)}
\subsection*{PPS is a research study that aims to discover if workplaces are a cause of idiopathic pulmonary fibrosis (IPF)}

The lead researcher is Dr Carl Reynolds, clinical research fellow at Imperial College London.

\section*{PART 1}
\subsection*{Can you help with a research study?}

\begin{itemize}
 \item We would like to invite you to take part in a research study. Before you decide we would like you to understand why the research is being carried out and what it would involve for you.  
 \item One of our team will go through this information sheet with you and answer any questions you have. This should take about 10--15 minutes.
 \item Please talk to others about the study if you wish and ask us if anything is not clear.
\end{itemize}

\subsection*{What is the purpose of the study?}
\begin{itemize}
 \item Idiopathic pulmonary fibrosis (also called IPF) is a disease that causes scarring of the lungs. The scarring damages the air sacs that allow oxygen to be transferred to the blood and transported to vital organs. IPF is a serious illness that causes cough, shortness of breath, and fatigue.
 \item We don't know what causes IPF but it is becoming more common in England, Scotland and Wales where it affects over 4000 people each year. People who get IPF are usually older than 40; the disease is more common in men and in parts of the country with a history of heavy industry.  
 \item This study will help to find out how much IPF can be attributed to workplace environments in England, Scotland and Wales. This will help us to better understand the causes of IPF, make sure people get the right treatment and compensation they are entitled to, and ensure that the controls at work are right so that we protect workers and prevent disease in the future.
\end{itemize}


\includegraphics[width=2.5cm]{fig/wellcome-logo-black.jpg}
\hspace{1cm}
\includegraphics[width=4cm]{fig/lungsatwork-logo.jpg}
\hspace{1cm}
\includegraphics[width=4cm]{fig/imperial-logo.jpg}

\subsection*{Why have I been chosen?}
\begin{itemize}
 \item The study works by comparing people with IPF (cases) to people who are similar but do not have IPF (controls). Both groups are essential for the study.
 \item You have been chosen to take part in the study as a \textbf{case} if you have a new diagnosis of IPF\@.
 \item You have been chosen to take part in the study as a \textbf{control} if you do not have IPF but recently had a hospital outpatient appointment and are of a similar age to patients who are newly diagnosed with IPF\@.  
\end{itemize}

\subsection*{Do I have to take part?}
\begin{itemize}
 \item It is up to you to decide if you want to take part in the research. We will describe the study and go through this information booklet with you.
 \item If you agree to take part we will ask you to read and sign a consent form.
 \item You are free to withdraw at any time, without giving a reason. This will not affect any of the care you receive.
\end{itemize}

 
\subsection*{Who are the researchers?}
The research will be conducted by a team based at Imperial College London, Imperial College Healthcare NHS Hospitals, and Sheffield Foundation Trust NHS Hospitals. The research is funded by the Wellcome Trust. The main investigators are:
\begin{itemize}
    \item Dr Carl Reynolds, Wellcome Trust Clinical Research Training Fellow, NHLI (Imperial College London). (Chief investigator)
 \item Professor Paul Cullinan, Professor, Honorary consultant physician (respiratory medicine). Occupational and Environmental Medicine, NHLI (Imperial College London), Royal Brompton Hospital, London. Joint appointment; tenured. (Co-Investigator)
 \item Dr Chris Barber, Consultant physician (respiratory medicine), Northern General Hospital, Sheffield. (Co-Investigator)
 \item Dr Sara De Matteis, Clinical Lecturer, NHLI (Imperial College London). (Co-Investigator)
\end{itemize}

\section*{PART 2}

\subsection*{What will happen to you if you take part?}
\begin{itemize}
 \item If you agree to take part the researcher will contact you to arrange a telephone interview at a time that is convenient for you.
 \item The telephone interview will last no longer than one hour.
 \item During the interview you will be asked questions about \begin{itemize}
                                                                \item All of the jobs you have had since leaving school; we may also ask about the jobs of people you have lived with
                                                                \item Your lifetime smoking history
                                                               \end{itemize}
\item You will be contacted to arrange a blood test to investigate genetic susceptibility to IPF\@. If possible the blood test will be taken when you next have blood tests to avoid an extra test. If this is
not possible it will be arranged at a time and place that is convenient for you. We will cover any reasonable travel expenses incurred due to participation in the study and agreed in advance.
\item With your permission, we will write to your GP to inform them that you are participating.
\item We will tell you what we find. What we find might not contain any helpful information for you. If we find anything we think is important we will, with your permission, inform your clinical team.
\end{itemize}

\subsection*{Why are you requesting a blood test?}

We want to know if workplace environments are a cause of IPF. We know that for most diseases whether or not a person gets the disease depends both on what they encounter in their environment, and the DNA or genes they are born with.

IPF is a rare disease. It is not a disease that normally runs in families but it is more common in people with certain genetic differences, such as a small change that affects mucus in our airways (called MUC5B rs35705950). The blood test helps us to check if it is workplace environments together with these genetic differences that causes IPF.   

\subsection*{What will the result of the blood test mean for me?}

If you are found to carry the MUC5B rs3570.10 genetic difference it does not mean that you have IPF or that you or your family members will get IPF. 

Studies have shown that you are about six times more likely to have IPF if you carry MUC5B rs3570.10. However, IPF is rare (fewer than one in 2000 people in the UK are diagnosed with the condition at some time in their life), so the overall risk of IPF for people who carry MUC5B rs3570.10 is still very low.    

\subsection*{Are there any benefits to taking part?}

It is unlikely that the study will help you personally. The information we get from this research may help to understand the causes of IPF, make sure people get the right treatment and compensation they are entitled to, and ensure that the controls on chemicals at work are right so that we protect workers and prevent disease in the future.

Patients with diseases that are discovered to be caused by work might get compensation. Currently, patients with IPF are unlikely to get compensation because it is not known to be caused by work. If we find that workplace environments do cause IPF for some people then this may change for patients in the future.   

\subsection*{Are there any risks to taking part?}

The greatest risk to you of participation in this study is an inadvertent disclosure of your private identifiable information. To minimize the risk of loss of confidentiality your interview response (and blood sample) will not be labelled with your private identifiable information. Interview response information will be kept encrypted on a computer in a locked office. Blood samples will be stored in a secure facility. You will not be identified in any report or publication of this study or its results.

There is a risk that we will find something that is important to your health. This could be distressing to you. If we find anything that we think could be important to your health we will inform you, and with your permission, your GP and hospital doctors. 

The study has been reviewed by the Nottingham 1 Research Ethics Committee.

\subsection*{What will happen when the research is finished?}

A summary of the results will be available and we will send you a copy if you request it. Data from the study, including anonymised unprocessed data, will be communicated to the wider academic community, and policy-makers, by publication and presentation at national and international respiratory and epidemiology meetings. Summary data will also be shared with the care teams participating in the study. 

\subsection*{What if there is a problem?}

If you have a concern about any aspect of this study, you should ask to speak to the researchers. They will do their best to answer your questions. Their contact details are on the last page of this booklet. If you remain unhappy and wish to complain formally you can do this by contacting the Patient Advice and Liaison Service (PALS). \\

\newpage

\begin{flushleft}

Patient Advice and Liaison Service (PALS) \\    
Ground floor of the Queen Elizabeth the Queen Mother (QEQM) building, \\
St Mary’s Hospital, \\
South Wharf Road,\\
London W2 1NY.\\
Tel: 020 3312 7777\\
Email: pals@imperial.nhs.uk\\

\end{flushleft}

Imperial College London holds insurance policies which apply to this study. If you experience serious and enduring harm or injury as a result of taking part in this study, you may be eligible to claim compensation without having to prove that Imperial College is at fault. This does not affect your legal rights to seek compensation.
 
If you are harmed due to someone’s negligence, then you may have grounds for a legal action. Regardless of this, if you wish to complain, or have any concerns about any aspect of the way you have been treated during the course of this study then you should immediately inform the Investigator (Carl Reynolds, contact details below). The normal National Health Service complaints mechanisms are also available to you. If you are still not satisfied with the response, you may contact the
Imperial AHSC Joint Research Compliance Office. 


\subsection*{What will happen to the information we collect?}

The Chief Investigator (Dr Carl Reynolds) will be responsible for ensuring that all the information we collect about you during the study is kept strictly confidential. For us to contact you it will be necessary your care team at the hospital to share your contact details with us. 
Any medical information about you which leaves the hospital/surgery will have your name and address removed so that you cannot be recognised from it. 

All the procedures used for handling, processing, storage and destruction of your information will be in compliance with the Data Protection Act 1998. All the information we collect will be encrypted and stored on a password protected computer in a secure building. Blood samples will be analyzed and stored in a secure lab at Imperial College London. 

Samples and data will be stored for 10 years after the study is finished. Only members of the research team will have access to the information collected and the ability to link it to you. Anonymised samples and data may be shared with academic units and any pharmaceutical collaborators.

\begin{centering}
\subsection*{Thank-you for your interest}
\subsection*{Please ask if you have questions}
\end{centering}

\vspace{1cm}

\paragraph{Contact} 
\begin{flushleft}
Dr Carl Reynolds / carl.reynolds@imperial.ac.uk / 07737 904 807 \\ 
National Heart and Lung Institute\\
Room G39 Emmanual Kaye Building\\
1b Manresa Road, London, SW3 6LR
\end{flushleft}


 
\end{document}
